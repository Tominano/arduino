\documentclass[a4paper,11pt]{article}
\usepackage[utf8]{inputenc}
\usepackage[T1]{fontenc}
\usepackage[magyar]{babel}
\usepackage{geometry}
\geometry{a4paper, tmargin = 18mm,
lmargin = 25mm, rmargin = 25 mm, bmargin = 18 mm}
\usepackage{mathtools}
\usepackage{listings}
\usepackage{multirow}
\usepackage{setspace}
\usepackage{graphicx}
\usepackage{wrapfig}
\usepackage{enumitem}

\usepackage{listings}
\lstset{language=C}

\author{Nagy Kapolcs Ompoly}
\title{Arduino Clap Sensitive Light Control}
\date{ }
\linespread{1.42}
\begin{document}
\begin{titlepage}
	\centering
	\includegraphics[width=0.15\textwidth]{eltecimer.jpg}\par
	\vspace{1cm}
	{\Large\itshape Mikrokontrollerek és alkalmazásaik Labor\par}
	{\huge\bfseries Arduino Clap Sensitive Light Control\par}
	
	\vfill
	
	\raggedleft
	Beadás: 2019.05.17.\par
	\vspace{0.5cm}
	Nagy Kapolcs Ompoly\par
	(W7R17G)\par
	3. évfolyam\par
	Pénteki csoport\par
	
	\vspace{0.5cm}

	\centering
	{\small\itshape Félévi Projekt Jegyzőkönyv \par}
\end{titlepage}
\clearpage
\setcounter{page}{1}
\newpage
\renewcommand{\thesection}{\Roman{section}}
\renewcommand{\thesubsection}{\thesection.\arabic{subsection}}
\renewcommand{\thesubsubsection}{\thesubsection.\arabic{subsubsection}}
\section{Projektmunka célja}
A projekt célja, hogy mikrokontroller segítségével egy LED szalagot írányítsunk hangérzékelővel, mivel így 1 vagy 2 kézen kívül nincs szükség a több végtagra, hogy tudjuk kontrollálni a környezet fényforrásának az állapotát.

\section{Eszközök}

\begin{itemize}
	\item Uno R3 board 
	\item USB cable
	\item Jump Wires
	\item Sound Sensor Module 
	\item SS8050 NPN Transistor
	\item LED strip
	\item 12v AC/DC LED Driver
\end{itemize}

\section{Projektmunka}

\subsection{Leírása}




\subsection{Felhasznált kód}

\begin{lstlisting}
int ledPin=9;
int sensorPin=7;

boolean val = 0; // sensor HIGH or LOW
boolean status_lights = false;

// for counting and calibrating clap
int clap = 0;
long detection_range_start = 0;
long detection_range = 0;

void setup(){
  pinMode(ledPin, OUTPUT);
  pinMode(sensorPin, INPUT);
  //Serial.begin(9600);
}
  
void loop (){
  int status_sensor = digitalRead(sensorPin);
  
  if (status_sensor == 1){
    if (clap == 0){
      detection_range_start = detection_range = millis();
      clap++; 
    } else if (clap > 0 && millis()-detection_range >= 100){
      detection_range = millis();
      clap++;
      //Serial.print("counting clap: ");
      //Serial.println(clap);
    }
  }
  
  if (millis()-detection_range_start >= 600){
    if (clap == 2){ 
      if (!status_lights){
        digitalWrite(ledPin, HIGH);
        status_lights = true;
        clap = 0;
      } else if (status_lights){
        status_lights = false;
        digitalWrite(ledPin, LOW);
      }
    }
    clap = 0;
  }
}
\end{lstlisting} 

\section{Mérési adatok és kiértékelés}

\subsection{Egy rés}

A méréshez tartozó grafikont csatolok a jegyzőkönyvhöz. Ezt a mérés további részeiben készített grafikonokat a mérőhelyen elhelyezett számítógép és kiértékelő program segítségével készítettük. \\
Az egyrésnél mért minimumok helyei.
\begin{center}
\begin{tabular}{|c|c|c|c|} \hline
\multicolumn{4}{|c|}{Rés elhajlási képének kioltási helyei} \\ \hline
k & $x_k$ & k & $x_k$ \\ \hline
-5 & 58.012 & 1 & 117.9347 \\ \hline
-4 & 67.8843 & 2 & 127.807 \\ \hline
-3 & 77.9862 & 3 & 137.2202 \\ \hline
-2 & 87.3994 & 4 & 147.0925 \\ \hline
-1 & 97.2717 & 5 & 156.9649 \\ \hline
\end{tabular}
\end{center}
Az adathalmazra egyenest illesztettem.
\begin{center}
#\includegraphics[width=0.66\textwidth]{1MINIMUM.png}
\end{center}

Az illesztett egyenes meredeksége:
\begin{center}
$m=(9.91\pm0.03)~mm$
\end{center}

A meredekség és a mért $L=(2067\pm1)~mm$ ernyőtávolság segítségével a rés szélessége:
\begin{center}
$a=\frac{\lambda L}{m}=(0.1319\pm0.0002)~mm$
\\
$\Delta a=a(\frac{\Delta \lambda}{\lambda}+\frac{\Delta L}{L}+\frac{\Delta m}{m})$
\end{center}

\subsection{Kéttős rés}

A csatolt grafikon a kettős rés elhajlási képének másodrendű minimumhelyeit ábrázolja a sorszámuk függvényében.
\begin{center}
\begin{tabular}{|c|c|c|c|} \hline
k & min & k & min \\ \hline
-3 & 73.5362 & 1 & 119.3754 \\ \hline
-2 & 85.0663 & 2 & 130.6243 \\ \hline
-1 & 96.4558 & 3 & 142.0138 \\ \hline
\end{tabular}
\end{center}
Az adathalmazra egyenest illesztettem.
\begin{center}
\includegraphics[width=0.66\textwidth]{2MINIMUM.png}
\end{center}
Az illesztett egyenes meredeksége:
\begin{center}
$m=(11.41\pm0.02)~mm$
\end{center}
A meredekség és a mért $L=(2067\pm1)~mm$ ernyőtávolság segítségével a két rés távolsága:
\begin{center}
$d=\frac{\lambda L}{m}=(0.1392\pm0.0003)~mm$
\\
$\Delta d=d(\frac{\Delta \lambda}{\lambda}+\frac{\Delta L}{L}+\frac{\Delta m}{m})$
\end{center}

\subsection{Hajszál}
A csatolt grafikon a hajszál elhajlási képének minimumhelyeit ábrázolja a sorszámuk függvényében.
\begin{center}
\begin{tabular}{|c|c|c|c|} \hline
k & min & k & min \\ \hline
-3 & 58.2044 & 1 & 125.244 \\ \hline
-2 & 75.1801 & 2 & 142.048 \\ \hline
-1 & 91.4611 & 3 & 159.032 \\ \hline
\end{tabular}
\end{center}
Az adathalmazra egyenest illesztettem.
\begin{center}
\includegraphics[width=0.66\textwidth]{3MINIMUM.png}
\end{center}
Az illesztett egyenes meredeksége:
\begin{center}
$m=(16.78\pm0.04)~mm$
\end{center}
A meredekség és a mért $L=(2067\pm1)~mm$ ernyőtávolság segítségével a hajszál szélessége:
\begin{center}
$a=\frac{\lambda L}{m}=(0.1080\pm0.0003)~mm$
\\
$\Delta a=a(\frac{\Delta \lambda}{\lambda}+\frac{\Delta L}{L}+\frac{\Delta m}{m})$
\end{center}

\subsection{Penge}
Itt analitikusan nem tudunk számolni, de a mérés során adatsorra illesztett elméleti görbét kinyomtatva csatolom a jegyzőkönyvhöz. \\
A mérés megkezdése előtt egy nyalábtágítót helyeztünk el a lézer és a penge közé, ezzel biztosítottuk a kellően nagy teret. 
\begin{center}
\includegraphics[width=0.66\textwidth]{freshnel.png}
\end{center}
\begin{center}
$a=633~mm$ és $b=2067.75~mm$
\end{center}
A féltér határát a penge éle határozza meg. Az elméleti és az illesztett görbe nem fedi teljesen egymást. Ennek oka az, hogy a penge éle nem tekinthető végtelen vékonynak, valamint a tér sem korlátlan.

\section{Diszkusszió}
A mérést sikeresnek tekinthetjük,az illesztések pontosak és a mérési eredmények hűen tükrözik a valóságot.
\end{document}